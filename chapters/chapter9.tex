\chapter{Conclusion}\label{chap:conclusion}

\section{Summary of Findings}

This research introduced and thoroughly evaluated the \acrfull{\acrshort{NOnt+SLabel}Label} ontology, a specialized framework designed to semantically represent geospatial knowledge within narrative contexts. The ontology demonstrated notable adaptability and extensibility by implementing \acrshort{NOnt+SLabel} across various interdisciplinary domains—such as bioeconomy and medieval geographical literature. The findings indicate that \acrshort{NOnt+SLabel} successfully models complex spatial and thematic relationships, thereby fulfilling the research objectives and establishing itself as a robust framework that unifies geospatial and narrative dimensions. One of the standout features of \acrshort{NOnt+SLabel} is its modular architecture, which allows for customization to meet the specific demands of diverse fields. For instance, in the bioeconomy, ontology facilitated the representation of value chains, while in historical geography, it provided tools for modelling intricate spatial-temporal relationships. This adaptability underlines the ontology’s potential as a comprehensive and flexible resource for managing geospatial narratives across disciplines.

\section{Contributions to Knowledge}

This study advances the field of geospatial narrative ontology, particularly in digital libraries, by developing the \acrshort{NOnt+SLabel} ontology — a framework that integrates narrative and spatial elements to model complex, layered narratives. \acrshort{NOnt+SLabel} innovatively combines narrative structures, such as \textit{fabula} (chronological story sequence) and \textit{syuzhet} (narrative presentation), with spatial constructs like qualitative and quantitative places. By bridging narrative and spatial dimensions, \acrshort{NOnt+SLabel} fills a critical gap in conventional ontologies, which often struggle to represent multifaceted narrative-spatial relationships. This framework thus sets a new standard for flexible and comprehensive modelling, applicable across diverse scientific domains that require sophisticated geospatial narrative representation.

To ensure interoperability and ease of integration, \acrshort{NOnt+SLabel} adopts established standards such as the CIDOC CRM, an ISO-standard vocabulary used extensively in cultural heritage. Extending this vocabulary promotes compatibility with existing ontological frameworks, enabling \acrshort{NOnt+SLabel} to integrate seamlessly with established data systems and legacy structures. Additionally, adherence to GeoSPARQL standards ensures that \acrshort{NOnt+SLabel} can handle complex geospatial queries while remaining compatible with semantic web technologies, further enhancing its adaptability across varied applications.

A pivotal feature of \acrshort{NOnt+SLabel} is its embedded semantic reasoner, which enables the inference of new knowledge based on existing data. This reasoning mechanism allows the ontology to uncover hidden patterns and relationships within narrative and spatial data, generating insights that may otherwise remain undetected. By enhancing knowledge depth and applicability, \acrshort{NOnt+SLabel} offers a valuable tool for research fields that require nuanced understanding of narrative-geospatial relationships. The ontology’s open science-oriented approach also supports transparency and broad accessibility, encouraging collaborative research and facilitating its adoption across the research community.

In addition to these structural and semantic advancements, this study developed a software architecture that visualizes knowledge through Story Maps. Story Maps provide an intuitive, narrative-driven interface for exploring geospatial narratives, integrating text, images, and maps into cohesive visual stories. This visualization capability enables a deeper, more immersive user experience, facilitating understanding of complex narrative and spatial interrelationships. Story Maps thus extend the utility of \acrshort{NOnt+SLabel} by offering an accessible, interactive platform that supports researchers, educators, and professionals in analyzing geospatial narratives.

This research introduces methodological innovations to support the construction of structured geospatial knowledge graphs, enabling more efficient data interoperability and visualization. By embedding reasoning mechanisms and aligning with Linked Open Data (LOD) principles, \acrshort{NOnt+SLabel} ensures data reusability and broad compatibility across fields. Its standards-compliant design positions \acrshort{NOnt+SLabel} as a robust, adaptable tool that supports interdisciplinary collaborations in areas such as digital humanities and bioeconomy, where complex geospatial narrative modeling is essential.

Validated across diverse research contexts, \acrshort{NOnt+SLabel} demonstrates effectiveness in managing intricate geospatial queries and representing narrative structures. Its versatility highlights its potential as a foundational tool within the semantic web landscape, serving as a significant asset for advanced geospatial narrative modeling and supporting the future of narrative representation across domains.

\section{Limitations}

While \acrshort{NOnt+SLabel} has proven to be robust and flexible, certain limitations emerged during its application in diverse domains. One notable limitation pertains to domain-specific scalability. Although the ontology’s modular structure provides a degree of flexibility, the need for specialized extensions to fully capture certain domain-specific complexities may impact scalability when applying the ontology across highly varied fields. This challenge underscores the need for ongoing refinement to ensure that the ontology remains adaptable without compromising its core structure or functionality.

Data integration presented another challenge, especially when dealing with heterogeneous data sources characterized by varying levels of granularity. Ensuring consistency and semantic coherence across these sources proved difficult, emphasizing the need for additional refinements to improve compatibility with diverse data structures. Furthermore, \acrshort{NOnt+SLabel} encountered limitations in representing certain aspects of geospatial data, particularly with respect to managing complex spatial-temporal relationships. While the ontology supports standard geospatial formats, constraints in dynamic temporal representation may restrict its effectiveness in applications involving rapidly evolving datasets. Addressing these limitations is essential for enhancing \acrshort{NOnt+SLabel}’s capability to operate seamlessly across dynamic and varied research contexts.

\section{Future Work}

In light of the contributions and limitations identified, several promising directions for future research are proposed. Enhancing the geospatial data modeling capabilities of \acrshort{NOnt+SLabel} is a primary focus. By integrating new geospatial data formats, such as GeoJSON, alongside advancements from GeoSPARQL 1.1 \cite{carGeoSPARQL11Motivations2022}, \acrshort{NOnt+SLabel} could achieve greater interoperability, broadening its applicability within dynamic data contexts. This integration would expand the ontology’s ability to represent a wider array of geospatial data types, making it more compatible with modern geospatial systems and increasing its relevance in diverse applications.

Another area for improvement is the integration of Natural Language Processing (NLP) modules. Automating the annotation process for geospatial narratives could significantly enhance scalability, allowing \acrshort{NOnt+SLabel} to manage larger, more diverse datasets with minimal manual intervention. This NLP integration would streamline data ingestion and enhance the ontology’s ability to handle comprehensive narrative datasets, further solidifying its role as a scalable and efficient tool for geospatial narrative representation.

Further interdisciplinary validation is also planned, particularly within the CRAEFT \cite{CraeftCareJudgment} and ECHOES \cite{EchoesEccchEuropean} projects. These projects aim to apply \acrshort{NOnt+SLabel} to represent geospatial knowledge related to traditional European craft and cultural heritage, respectively. Such applications will extend the ontology’s interdisciplinary relevance, reinforcing its capability to meet the demands of various research domains. This planned validation across different fields will not only improve the ontology’s practical applicability but also deepen its theoretical grounding, thereby broadening its scope for future use.

\section{Conclusion}

In conclusion, the evaluation and validation of the \acrshort{NOnt+SLabel} ontology have demonstrated its applicability across multiple domains and its potential for generating new insights through semantic reasoning. By bridging narrative and geospatial knowledge representation, \acrshort{NOnt+SLabel} offers a robust framework capable of handling complex spatial narratives. While there are areas for refinement—particularly concerning scalability and data integration—the ontology’s contributions underscore its significant value to interdisciplinary research. The planned expansions of the ontology, including the incorporation of additional geographic formats and ongoing use in interdisciplinary projects, indicate a promising future. \acrshort{NOnt+SLabel} stands as a versatile tool, offering substantial contributions to the semantic web landscape by enabling the representation and analysis of geospatial knowledge across a wide range of research fields. With continued development and interdisciplinary applications, \acrshort{NOnt+SLabel} has the potential to become a foundational resource within the domain of geospatial knowledge representation, supporting more comprehensive and nuanced approaches to digital knowledge management.
