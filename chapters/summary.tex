\chapter*{Summary}

\lettrine{G}{eospatial} knowledge plays a fundamental role in shaping the understanding and communication of narratives. Whether historical, literary, or journalistic, narratives often incorporate spatial elements that provide essential context, enriching the reader’s engagement with the story. This thesis addresses a fundamental challenge in the fields of knowledge representation and the Semantic Web: the effective semantic modeling of geospatial narratives. Geospatial narratives are conceived as networks of events interconnected through semantic links, capturing the complex relationships between spatial occurrences. Current models lack sufficient expressiveness and interoperability, which limits their capacity for data integration, querying, and analysis. To overcome these limitations, this research develops the Geospatial Narrative Ontology (NOnt+S), an extension of the existing Narrative Ontology (NOnt) that is based on the ISO standard CIDOC Conceptual Reference Model (CIDOC CRM). By formally modeling geospatial information using a formal language like OWL 2 DL, the ontology ensures decidability and enhances interoperability across diverse knowledge bases. Furthermore, implementing this ontology within a Semantic Web framework and employing semantic reasoners enable the inference of new geospatial knowledge, thereby enriching data analysis and facilitating more sophisticated queries. Finally, the thesis addresses the challenges of visualization by proposing Story Maps as a solution to visualize geospatial narratives. Story Maps are interactive digital maps that tell stories by integrating geospatial data with narrative elements. Story mapping, in this context, is a method for arranging narrative events to create a more holistic view of how they fit into the overall geospatial narrative, thereby enhancing user engagement and understanding.

\section*{Research Objectives}

The thesis aims to advance geospatial knowledge representation in narratives by:
\begin{enumerate}
    \item Developing an ontology, NOnt+S, wich representan geospatial and temporal aspects of narratives ensuring interoperability with existing Semantic Web frameworks.
    \item Enhancing data integration and enabling advanced geospatial querying mechanisms to achieve consistent and comprehensive representations of geospatial information in narratives.
    \item Evaluating the ontology's practical implications and applicability, including its ability to infer new knowledge, through empirical validation in real-world case studies from the MOVING (Horizon 2020-2024) and IMAGO (PRIN 2020-2024) projects.
    \item Developing effective visualization methods for geospatial narratives by employing Story Maps, integrating geospatial data with narrative elements to enhance user understanding and engagement.
\end{enumerate}

\section*{Methodology}

The research employs a rigorous, structured methodology based on the METHONTOLOGY framework, which includes:
\begin{itemize}
    \item \textbf{Requirements Specification}: Identifying the requirements for geospatial narrative representation through comprehensive analysis.
    \item \textbf{Conceptualization and Logical Formalization}: Extending the NOnt ontology to incorporate geospatial and temporal constructs, providing formal mathematical specifications.
    \item \textbf{Implementation}: Developing NOnt+S in OWL 2 DL, ensuring decidability and interoperability to achieve syntactic precision and compatibility within the Semantic Web environment.
    \item \textbf{Evaluation}: Validating the ontology through case studies, assessing its performance in terms of semantic enrichment, reasoning capabilities, and querying effectiveness.
\end{itemize}

\section*{Significance of Findings}
This thesis advances Knowledge Representation by introducing the NOnt+S ontology, a robust framework for modeling complex geospatial narrative structures. NOnt+S integrates narrative elements like fabula (chronological sequence) and plot (narrative presentation) with spatial constructs such as qualitative and quantitavce place. By bridging narrative and spatial dimensions, it addresses limitations in existing ontologies and sets a new standard for flexible modeling applicable in digital humanities, GIS, and smart city applications.

Adopting established standards like the CIDOC CRM and GeoSPARQL ensures interoperability with existing data systems and semantic web technologies. An embedded semantic reasoner allows NOnt+S to infer new knowledge from existing data, uncovering hidden patterns within narrative and spatial information.

The study also developed a software architecture utilizing Story Maps, providing an intuitive interface for visualizing geospatial narratives by integrating text, images, and maps. By supporting the construction of structured geospatial knowledge graphs and aligning with Linked Open Data principles, NOnt+S promotes data reusability and interdisciplinary collaboration. Validated across diverse domains, it demonstrates effectiveness in managing intricate geospatial queries and serves as a significant asset for advanced geospatial narrative modeling, laying the groundwork for future advancements.

