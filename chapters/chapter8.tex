% \chapter{Discussion}

% \section{Overview of Contributions}

% This research advances the formal representation of geospatial knowledge in narratives through the development of the NOnt+S ontology, a geospatially extended framework built on the foundation of the existing Narrative Ontology (NOnt). The following discussion elaborates on the main contributions, strengths, and limitations of NOnt+S and explores the impact and future directions of this research.

% The primary achievement of this work lies in bridging the gap between the nuanced, narrative representation of geospatial elements and the need for structured, machine-readable data formats. Traditional ontologies have often struggled to accommodate the temporal and dynamic aspects of narratives, particularly when dealing with the inherent ambiguity and subjectivity of natural language. By integrating geospatial reasoning capabilities and adopting Semantic Web technologies, NOnt+S provides a robust solution to these challenges.

% \section{Analysis of Key Findings}

% The NOnt+S ontology is designed to model complex relationships between narrative events and spatial locations, facilitating efficient querying and reasoning. By extending NOnt to include geospatial elements, NOnt+S enhances the ability to represent both qualitative and quantitative spatial relationships. The introduction of geospatial reasoning, coupled with semantic reasoning, enables the inference of new spatial-temporal insights from narrative data. This dual capability represents a significant step forward in the computational analysis of narratives.

% \subsection{Effectiveness in Real-World Applications}

% The validation of NOnt+S through case studies in bioeconomy research and the study of medieval geographic literature demonstrates the ontology’s versatility. In the bioeconomy domain, the ontology's ability to enrich data with geospatial context enhances understanding of mountain ecosystems and facilitates interdisciplinary knowledge sharing. Similarly, in the cultural heritage domain, NOnt+S provides a structured way to explore complex narrative journeys, such as those depicted in medieval manuscripts, revealing patterns and connections that were previously difficult to identify.

% Furthermore, the visualization tool, Story Maps, developed as part of this research, adds a significant dimension to narrative exploration. By representing temporal progression alongside spatial data, Story Maps make abstract relationships tangible and intuitive, supporting educational and research endeavors in digital humanities and beyond.

% \section{Comparative Strengths and Limitations}

% \subsection{Strengths}

% One of the major strengths of NOnt+S is its adherence to Semantic Web standards, which ensures interoperability and facilitates integration with other ontologies. The use of OWL2 DL ensures logical consistency, while the incorporation of established frameworks like CIDOC CRM and GeoSPARQL enhances the ontology's applicability across multiple domains. The ontology’s design also emphasizes flexibility, allowing for future extensions to address domain-specific requirements.

% The semantic reasoner and geospatial reasoning mechanisms embedded within NOnt+S significantly enhance its analytical capabilities. These features enable the ontology to infer implicit knowledge, such as identifying recurring spatial patterns or understanding the movement of characters within a narrative. This dual-layered reasoning capability distinguishes NOnt+S from traditional narrative ontologies, which often lack comprehensive spatial reasoning.

% \subsection{Limitations}

% Despite these strengths, NOnt+S does face certain limitations. Scalability remains a concern, as extending the ontology to cover a wide array of domains may require significant customization. The integration of heterogeneous data sources also presents challenges; data from different origins often lack semantic coherence and may require extensive pre-processing. Furthermore, the ontology's current representation of spatiotemporal relationships, though advanced, struggles with highly dynamic datasets, and computational complexity increases significantly with large-scale narratives.

% The visualization component, while effective, may not fully capture the intricacies of complex spatial narratives. Although Story Maps provide an interactive and engaging means of exploration, the interpretive nature of narratives means that some spatial relationships may be oversimplified or misrepresented.

% \section{Implications for Future Research}

% The limitations identified in this study offer opportunities for future research. Addressing the ontology's scalability will require developing specialized extensions tailored to diverse domains, as well as optimizing data integration workflows. Further, enhancing NOnt+S's temporal reasoning capabilities, such as refining temporal primitives and exploring more efficient spatiotemporal representation methods, is a priority for future iterations.

% Incorporating Natural Language Processing (NLP) modules for automated narrative extraction could significantly improve the ontology's usability, particularly in dealing with unstructured text. By integrating tools that can parse and annotate narratives with geospatial tags, NOnt+S could automate much of the data pre-processing currently required. Additionally, exploring new data formats, such as GeoJSON and the forthcoming updates to GeoSPARQL 1.1, could improve the ontology's compatibility and ease of use.

% \section{Conclusion}

% In summary, NOnt+S represents a comprehensive advancement in the field of geospatial narrative representation. By addressing the limitations of traditional ontologies and offering new avenues for narrative exploration through Semantic Web technologies, this research lays the groundwork for future developments in both the digital humanities and scientific domains. While challenges remain, the potential for further refinement and application of this ontology is vast, paving the way for more sophisticated, integrated approaches to understanding and visualizing geospatial knowledge in narratives.
