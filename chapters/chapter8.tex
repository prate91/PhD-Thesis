
% \chapter*{Comparative Analysis of NOnt+S with Existing Ontologies and Representation Techniques}

% \section*{Introduction}
% The representation of narratives and their associated geospatial and temporal dimensions has emerged as a critical concern within knowledge modeling and digital humanities. The NOnt+S ontology, as an extension of the Narrative Ontology (NOnt), builds upon existing frameworks to address the challenges of integrating spatial semantics into narrative representation. Developed using OWL 2 DL, NOnt+S ensures compatibility with the Semantic Web and enhances reasoning capabilities. This chapter provides a thorough comparative analysis of NOnt+S against existing ontologies and knowledge representation approaches, including CIDOC CRM, narrative cartography, knowledge graphs, interactive platforms, early artificial intelligence techniques, and Fabio Ciotti's formal ontology for narrative. Each comparison highlights the unique contributions of NOnt+S in addressing the limitations of earlier models.

% \section*{Comparative Table}
% The following table summarizes the key features, strengths, and limitations of NOnt+S in comparison with existing ontologies and approaches:

% \begin{table}[h!]
% \centering
% \caption{Comparison of NOnt+S with Existing Ontologies and Techniques}
% \label{tab:comparison}
% \begin{tabular}{|>{\raggedright}p{4cm}|>{\raggedright}p{4cm}|>{\raggedright}p{4cm}|>{\raggedright\arraybackslash}p{4cm}|}
% \hline
% \textbf{Ontology/Technique} & \textbf{Focus} & \textbf{Strengths} & \textbf{Limitations Overcome by NOnt+S} \\
% \hline
% CIDOC CRM & Event-based cultural heritage & Robust event and relationship modeling & Lacks explicit geospatial semantics and reasoning capabilities. \\
% \hline
% Narrative Cartography & Visual storytelling through maps & Spatio-temporal visualization of narratives & Static; lacks computational reasoning and semantic formalization. \\
% \hline
% Knowledge Graphs & Interlinked multi-domain data & Integrates entities and relationships across domains & Limited reasoning capabilities and absence of geospatial constructs. \\
% \hline
% Interactive Platforms (e.g., InTaVia) & Post-anthropocentric storytelling & Supports multiple narrative entities and visualization & No formal reasoning; lacks geospatial querying and semantic interoperability. \\
% \hline
% Early AI Techniques (SEBNET, SEB) & Story grammars and relationships & Rule-based representation of characters and narratives & Limited scalability and semantic reasoning; no geospatial support. \\
% \hline
% Ciotti's Formal Ontology & Narrative components: characters, traits, and spaces & Formal OWL-based representation of narrative semantics & Focused on textual abstraction; lacks spatial semantics and reasoning. \\
% \hline
% \textbf{NOnt+S} & Narrative, geospatial, and temporal representation & Integrates geospatial semantics, formal reasoning, and advanced querying & Combines strengths of other approaches while addressing their limitations. \\
% \hline
% \end{tabular}
% \end{table}

% \section*{NOnt+S and CIDOC CRM}
% CIDOC CRM is a widely adopted standard for cultural heritage representation, with a particular emphasis on events and relationships between entities. Its robust structure facilitates the organization of temporal and historical knowledge, offering a foundational framework for mapping cultural narratives. However, while CIDOC CRM provides a strong basis for event-based data modeling, it lacks the expressiveness required to represent the fine-grained spatial semantics of geospatial narratives.

% In this context, NOnt+S extends the strengths of CIDOC CRM by formalizing geospatial knowledge explicitly. Through the integration of GeoSPARQL and OWL 2 DL, NOnt+S ensures that spatial relationships are not only captured but also reasoned upon. For instance, qualitative and quantitative spatial descriptors, such as proximity, topological relationships, and distance, become computable entities within the narrative structure. Thus, NOnt+S advances beyond CIDOC CRM by incorporating geospatial reasoning into narrative modeling, allowing for richer and more sophisticated querying and inference.

% \section*{Narrative Cartography and NOnt+S}
% Narrative cartography, as conceptualized by scholars like Caquard and Cartwright, emphasizes the dynamic interplay between maps and stories. Maps are frequently employed to represent the spatial and temporal dimensions of narratives, facilitating the visualization of story progressions, emotional geographies, and ambiguous spaces. However, traditional cartographic methods often remain static, relying on manual integration of spatial data without leveraging the computational reasoning that ontological approaches afford.

% NOnt+S overcomes this limitation by embedding narrative cartography within a formalized, semantic framework. Instead of relying solely on static visualizations, NOnt+S encodes spatial relationships and narrative events into a machine-readable format. This formalization enhances automation, ensuring that the spatial semantics of narratives can be queried, reasoned, and visualized dynamically. The integration of OWL 2 DL enables inferential processes that reveal hidden patterns in spatial narratives, an advancement that narrative cartography, in its current form, cannot achieve.

% \section*{Knowledge Graphs and NOnt+S}
% The proliferation of knowledge graphs in digital humanities and geospatial studies has significantly improved the ability to integrate multi-domain data. Knowledge graphs, such as Wikidata and DBpedia, interlink entities and their relationships, offering a versatile platform for modeling complex systems of knowledge. For narrative studies, knowledge graphs have enabled the creation of interlinked story data, connecting people, events, places, and objects within a structured graph representation. Despite these advancements, traditional knowledge graphs often lack formal reasoning capabilities and explicit geospatial constructs.

% NOnt+S builds upon the strengths of knowledge graphs while addressing their limitations. By leveraging Semantic Web technologies such as GeoSPARQL, NOnt+S incorporates rigorous spatial semantics into its structure, enabling the representation of both qualitative and quantitative geospatial relationships. OWL 2 DL enhances this framework with logical reasoning, allowing for the inference of new relationships from existing data. Consequently, NOnt+S provides a more powerful tool for geospatial narrative analysis, surpassing the static, query-limited nature of traditional knowledge graphs.

% \section*{Conclusion}
% NOnt+S advances the state of narrative modeling by combining semantic reasoning, geospatial querying, and narrative formalization into a unified framework. Unlike earlier approaches, such as CIDOC CRM, knowledge graphs, and narrative cartography, NOnt+S explicitly integrates spatial semantics, ensuring that geospatial relationships can be formally represented and reasoned upon. While platforms like InTaVia prioritize visualization and interaction, NOnt+S provides the formal structure necessary for computational querying and inference. By incorporating spatial semantics, NOnt+S builds upon Fabio Ciotti's formal ontology, extending its narrative modeling capabilities into the geospatial domain. This ontology sets a new standard for narrative knowledge representation, bridging the gap between visualization, reasoning, and semantic interoperability.

% \chapter{Discussion}

% \section{Overview of Contributions}

% This research advances the formal representation of geospatial knowledge in narratives through the development of the NOnt+S ontology, a geospatially extended framework built on the foundation of the existing Narrative Ontology (NOnt). The following discussion elaborates on the main contributions, strengths, and limitations of NOnt+S and explores the impact and future directions of this research.

% The primary achievement of this work lies in bridging the gap between the nuanced, narrative representation of geospatial elements and the need for structured, machine-readable data formats. Traditional ontologies have often struggled to accommodate the temporal and dynamic aspects of narratives, particularly when dealing with the inherent ambiguity and subjectivity of natural language. By integrating geospatial reasoning capabilities and adopting Semantic Web technologies, NOnt+S provides a robust solution to these challenges.

% \section{Analysis of Key Findings}

% The NOnt+S ontology is designed to model complex relationships between narrative events and spatial locations, facilitating efficient querying and reasoning. By extending NOnt to include geospatial elements, NOnt+S enhances the ability to represent both qualitative and quantitative spatial relationships. The introduction of geospatial reasoning, coupled with semantic reasoning, enables the inference of new spatial-temporal insights from narrative data. This dual capability represents a significant step forward in the computational analysis of narratives.

% \subsection{Effectiveness in Real-World Applications}

% The validation of NOnt+S through case studies in bioeconomy research and the study of medieval geographic literature demonstrates the ontology’s versatility. In the bioeconomy domain, the ontology's ability to enrich data with geospatial context enhances understanding of mountain ecosystems and facilitates interdisciplinary knowledge sharing. Similarly, in the cultural heritage domain, NOnt+S provides a structured way to explore complex narrative journeys, such as those depicted in medieval manuscripts, revealing patterns and connections that were previously difficult to identify.

% Furthermore, the visualization tool, Story Maps, developed as part of this research, adds a significant dimension to narrative exploration. By representing temporal progression alongside spatial data, Story Maps make abstract relationships tangible and intuitive, supporting educational and research endeavors in digital humanities and beyond.

% \section{Comparative Strengths and Limitations}

% \subsection{Strengths}

% One of the major strengths of NOnt+S is its adherence to Semantic Web standards, which ensures interoperability and facilitates integration with other ontologies. The use of OWL2 DL ensures logical consistency, while the incorporation of established frameworks like CIDOC CRM and GeoSPARQL enhances the ontology's applicability across multiple domains. The ontology’s design also emphasizes flexibility, allowing for future extensions to address domain-specific requirements.

% The semantic reasoner and geospatial reasoning mechanisms embedded within NOnt+S significantly enhance its analytical capabilities. These features enable the ontology to infer implicit knowledge, such as identifying recurring spatial patterns or understanding the movement of characters within a narrative. This dual-layered reasoning capability distinguishes NOnt+S from traditional narrative ontologies, which often lack comprehensive spatial reasoning.

% \subsection{Limitations}

% Despite these strengths, NOnt+S does face certain limitations. Scalability remains a concern, as extending the ontology to cover a wide array of domains may require significant customization. The integration of heterogeneous data sources also presents challenges; data from different origins often lack semantic coherence and may require extensive pre-processing. Furthermore, the ontology's current representation of spatiotemporal relationships, though advanced, struggles with highly dynamic datasets, and computational complexity increases significantly with large-scale narratives.

% The visualization component, while effective, may not fully capture the intricacies of complex spatial narratives. Although Story Maps provide an interactive and engaging means of exploration, the interpretive nature of narratives means that some spatial relationships may be oversimplified or misrepresented.

% \section{Implications for Future Research}

% The limitations identified in this study offer opportunities for future research. Addressing the ontology's scalability will require developing specialized extensions tailored to diverse domains, as well as optimizing data integration workflows. Further, enhancing NOnt+S's temporal reasoning capabilities, such as refining temporal primitives and exploring more efficient spatiotemporal representation methods, is a priority for future iterations.

% Incorporating Natural Language Processing (NLP) modules for automated narrative extraction could significantly improve the ontology's usability, particularly in dealing with unstructured text. By integrating tools that can parse and annotate narratives with geospatial tags, NOnt+S could automate much of the data pre-processing currently required. Additionally, exploring new data formats, such as GeoJSON and the forthcoming updates to GeoSPARQL 1.1, could improve the ontology's compatibility and ease of use.

% \section{Conclusion}

% In summary, NOnt+S represents a comprehensive advancement in the field of geospatial narrative representation. By addressing the limitations of traditional ontologies and offering new avenues for narrative exploration through Semantic Web technologies, this research lays the groundwork for future developments in both the digital humanities and scientific domains. While challenges remain, the potential for further refinement and application of this ontology is vast, paving the way for more sophisticated, integrated approaches to understanding and visualizing geospatial knowledge in narratives.
