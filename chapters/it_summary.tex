\chapter*{Sommario}
\lettrine{L}{a} conoscenza geospaziale ha un impatto significativo sulla comprensione e sulla comunicazione delle storie. Le narrazioni — che siano di natura storica, letteraria o giornalistica — spesso includono elementi spaziali che offrono un contesto fondamentale per la comprensione del racconto. Questa tesi affronta una sfida fondamentale nei campi della rappresentazione della conoscenza e del Web Semantico: la modellazione semantica delle narrazioni geospaziali. Le narrazioni geospaziali sono concepite come reti di eventi interconnessi attraverso legami semantici, catturando le complesse relazioni tra le occorrenze spaziali. I modelli attuali mancano di sufficiente espressività e interoperabilità, il che limita la loro capacità di integrazione dei dati, interrogazione e analisi. Per superare queste limitazioni, questa ricerca sviluppa l'Ontologia delle Narrazioni Geospaziali (NOnt+S), un'estensione dell'esistente Ontologia delle Narrazioni (NOnt) che si basa sullo standard ISO CIDOC Conceptual Reference Model (CIDOC CRM). La modellazione formale delle informazioni geospaziali tramite un linguaggio formale come OWL 2 DL garantisce la decidibilità e migliora l'interoperabilità tra diverse basi di conoscenza. Inoltre, implementando questa ontologia all'interno di un framework del Web Semantico e impiegando ragionatori semantici, si rende possibile l'inferenza di nuova conoscenza geospaziale, arricchendo così l'analisi dei dati e facilitando interrogazioni più sofisticate. Infine, la tesi affronta le sfida di una visualizzazione efficace proponendo le \textit{Story Maps} come soluzione per visualizzare le narrazioni geospaziali. Le \textit{Story Maps} sono mappe digitali interattive che raccontano storie integrando dati geospaziali con elementi narrativi. Lo \textit{story mapping}, in questo contesto, è un metodo per organizzare gli eventi narrativi al fine di creare una visione più olistica di come essi si inseriscono nella narrazione geospaziale complessiva, migliorando così il coinvolgimento e la comprensione dell'utente.

\section*{Obiettivi della ricerca}

La tesi mira ad avanzare la rappresentazione della conoscenza geospaziale nelle narrazioni proponendosi di:

\begin{enumerate}
    \item Sviluppare un'ontologia, NOnt+S, che rappresenta gli aspetti geospaziali e temporali delle narrazioni, garantendo l'interoperabilità con i framework esistenti del Web Semantico.
    \item Migliorare l'integrazione dei dati e abilitare sitemi di interrogazione geospaziale per ottenere rappresentazioni coerenti e complete delle informazioni geospaziali nelle narrazioni
    \item Valutare le implicazioni pratiche e l'applicabilità dell'ontologia, inclusa la sua capacità di inferire nuova conoscenza, attraverso la validazione empirica in casi di studio reali dai progetti MOVING (Horizon 2020-2024) e IMAGO (PRIN 2020-2024).
    \item Sviluppare metodi efficaci di visualizzazione per le narrazioni geospaziali utilizzando Story Maps, integrando dati geospaziali con elementi narrativi per migliorare la comprensione e il coinvolgimento dell'utente.
\end{enumerate}

\section*{Metodologia}

La ricerca impiega una metodologia rigorosa e strutturata basata sul framework METHONTOLOGY, che include:

\begin{itemize}
    \item \textbf{Specifica dei requisiti}: identificazione dei requisiti per la rappresentazione delle narrazioni geospaziali attraverso un'analisi esaustiva.
    \item \textbf{Concettualizzazione e formalizzazione logica}: estendere l'ontologia NOnt per incorporare costrutti geospaziali e temporali, fornendo una formalizzazione matematica.
    \item \textbf{Implementazione}: sviluppare NOnt+S in OWL 2 DL, garantendo decidibilità e interoperabilità per ottenere precisione sintattica e interoperabilità all'interno del Web Semantico.
    \item \textbf{Valutazione}: validare l'ontologia attraverso casi di studio, valutando le sue prestazioni in termini di arricchimento semantico, capacità di ragionamento ed efficacia nelle interrogazioni.
\end{itemize}

\section*{Importanza dei risultati}

Questa tesi avanza la rappresentazione della conoscenza introducendo l'ontologia NOnt+S, un framework robusto per modellare narrazioni geospaziali complesse. NOnt+S integra elementi narrativi come la fabula (sequenza cronologica) e l'intreccio (presentazione narrativa) con costrutti spaziali quali i luoghi qualitativi e quantitativi. Collegando le dimensioni narrative e spaziali, affronta le limitazioni delle ontologie esistenti e stabilisce un nuovo standard per la modellazione flessibile applicabile nelle \textit{digital humanities}, nei GIS e nelle applicazioni per le smart city.

Adottando standard consolidati come il CIDOC CCnceptual Reference Model (CIDOC CRM) e GeoSPARQL, si garantisce l'interoperabilità con i dati esistenti e le tecnologie del Web Semantico. Tramite un ragionatore semantico integrato, NOnt+S è in grado di inferire nuova conoscenza dai dati esistenti, rivelando pattern nascosti all'interno delle informazioni narrative e geospaziali.

Lo studio ha inoltre sviluppato un \textit{framework} software che utilizza Story Maps, fornendo un'interfaccia intuitiva per visualizzare narrazioni geospaziali attraverso l'integrazione di testo, immagini e mappe. Supportando la costruzione di grafi di conoscenza geospaziale strutturati e allineandosi ai principi dei Linked Open Data (LOD), NOnt+S promuove la riutilizzabilità dei dati e la collaborazione interdisciplinare. Validata in domini diversi, dimostra efficacia nella gestione di interrogazioni geospaziali complesse e rappresenta una risorsa significativa per la modellazione avanzata di narrazioni geospaziali, gettando le basi per futuri progressi.